\documentclass[11pt]{article}
%\documentclass[3p,review,11pt]{elsarticle}
\usepackage{lineno,hyperref,notoccite,etoolbox}
\modulolinenumbers[5]
\usepackage[margin=1in]{geometry}

\makeatletter
\def\ps@pprintTitle{%
	\let\@oddhead\@empty
	\let\@evenhead\@empty
	\def\@oddfoot{\centerline{\thepage}}%
	\let\@evenfoot\@oddfoot}
\makeatother
\usepackage{setspace}
\singlespacing
\usepackage{mathptmx}
\usepackage{float,wrapfig}

\usepackage{sectsty}
\sectionfont{\fontsize{12}{15}\selectfont}
\newcommand{\vs}{\vspace{2mm}}
\begin{document}

\begin{center}
Understanding and Mitigating Uranium Nitride Corrosion for Use in Nuclear Reactors \par \vs
Ember Sikorski 
\section*{Abstract}
\end{center}

\rule{6.2in}{0.1mm}
\vs \par 
Uranium nitride offers higher thermal conductivity, higher melting temperature, and higher actinide density than uranium dioxide. Despite these benefits, uranium nitride is not as stable as uranium dioxide in air and readily degrades. Current research on the degradation of UN pellets, powders, and films is reviewed. Studies suggest nitriding surfaces and adding intermetallic phases may improve uranium nitride performance. Computational studies show atomistically stable configurations of adsorbates at uranium nitride surfaces. \par 
\rule{6.2in}{0.1mm}	
\vs


\section{Introduction}
%introduce topic
%highlight importance of topic to nuclear power production
%materials it affects
%components of the reactor involved

Uranium nitride (UN) is considered a prospective fuel for both light water reactors (LWRs) and Generation IV reactors \cite{Streit2005,Mizutani1998}. The higher fissile density of UN as compared to uranium dioxide benefits fast reactors given the lower neutron cross section \cite{Silva2009}. This higher actinide density additionally benefits LWRs because UN pellets can remain in the reactor longer, leading to longer time between shut downs, and reducing money lost \cite{Lopes2017}. For all reactor types, the high thermal conductivity and high melting temperature make UN an optimal material to resist accidents \cite{Lopes2017}.
\par 
While nuclear power initially developed uranium dioxide for the fuel pellet, the call for accident tolerant fuels (ATF) after Fukushima has drawn attention back to nitrides \cite{Johnson2016}. However, ATF materials must maintain current operational standards as well as improve safety \cite{Zinkle2014}. Despite the beneficial properties of UN, it is unstable the presence of steam or even air \cite{Johnson2016,Jolkkonen2017,Lopes2017}. In the event of cladding failure in a LWR, the pellet will come into contact with steam. As UN degrades, fission products will be released from the fuel matrix, free to interact with the containment structure. Furthermore, nitrogen can react with the steam to form explosive ammonium nitrate \cite{Jolkkonen2017}.
\par 
 In the case of Generation IV reactors, most designs use alternative coolants. However, the instability of UN in air hinders fabrication into a fuel pellet \cite{Lopes2017}. Short of solving UN corrosion at high temperatures and pressures for LWRs, corrosion in air must be mitigated for use as a Generation IV fuel.
 \par 
 To assess the current state of UN corrosion research, this review examines six studies from within the past decade. The first two studies work to better understand UN oxidation in the presence of air \cite{Johnson2016} and steam \cite{Jolkkonen2017}. Next, Lu et al. \cite{Lu2016} and Lopes et al. \cite{Lopes2017} each investigated a method to mitigate UN corrosion: nitriding and introduction of an intermetallic phase, respectively. The last two studies use computational modeling to probe the atomistic initiation of UN corrosion, studying oxygen \cite{Bocharov2013} and water \cite{Bo2016} at UN surfaces.

\section{Review}
%current efforts in the area
%may include research on demonstration of issue or mitigation strategies


\par 
To better understand uranium nitride corrosion, Jolkkonen et al. \cite{Jolkkonen2017} subjected UN pellets to superheated steam. The authors tried varied densities of UN pellets, ranging from 77.6\% to 97.7\% of theoretical density. The lowest density pellet resulted in the greatest amount of degradation, yielding rapid H$_{2}$ production followed by a significant release of N$_{2}$ uncharacteristic of the other, denser pellets. The denser pellets could last up to 90 minutes while subjected to 300 $^{\circ}$C steam, but degraded as the temperature was increased to 400 $^{\circ}$C. The pellets showed different mechanical fracture modes, with the lowest density pellets turning to powder and the higher density pellets cracking into fragments. This fragmentation would further accelerate corrosion as additional surface is exposed.
\par 
Though the UN corrosion mechanism was first proposed as 

\begin{equation}
\mbox{UN} + 2\mbox{H}_{2}\mbox{O} \rightarrow \mbox{UO}_{2} + \mbox{NH}_{3} + \frac{1}{2} \mbox{H}_{2} \quad ,
\end{equation}
Jolkkonen et al. found U$_{2}$N$_{3}$ formation, which disrupts the stoichiometry of (1) by reducing NH$_{3}$ formation and increasing H$_{2}$ formation. Additionally, even after hydrolysis finished, residual N$_{2}$ was found in the resultant powder.

Johnson et al. \cite{Johnson2016} studied UN powders, uranium silicide, and uranium dioxide powders and compared the time until oxidation. The powders were ramped to 800$^{\circ}$C while the mass was measured with thermogravimetry. At 4.4\% open porosity, UN oxidized at 320 $^{\circ}$C, determined by 5\% mass increase. On the other hand, 0.0\% open porosity UN performed better than even UO$_{2}$, with oxidation onsets of 440 $^{\circ}$C and 405 $^{\circ}$C, respectively. The UN-10\%U$_{3}$Si$_{2}$ powder performed nearly the same as the low porosity UN, with a slightly higher oxidation onset temperature of 450$^{\circ}$C but also slightly higher mass increase due to oxide formation. The authors conclude that while using U$_{3}$Si$_{2}$ in conjunction with UN may improve fabrication of UN, it does not necessarily reduce corrosion.
\par 
Lopes et al. \cite{Lopes2017} took the work of Johnson et al. a step further by fabricating full pellets of UN and UN-10\%U$_{3}$Si$_{2}$. In order to reach the high densities sufficient to reduce oxidation, UN must be sintered at over 2000 $^{\circ}$C. However, this results in accelerated grain growth, creating fast diffusion pathways for both oxygen to enter and fission products to escape. To reduce this necessary temperature, an intermetallic phase of U$_{3}$Si$_{2}$ can be added during sintering.
\par 
Lopes et al. placed the pellets in an autoclave, heated to 300 $^{\circ}$C, and added 9 MPa steam. Like Jolkkonen et al. and Johnson et al., the authors found reduced reactivity with reduced porosity. They found degradation increased in time, and suggested the mechanism depended on the OH and NH$_{3}$ formed during the reaction. The oxygen formed a non-protective layer and larger grains were easily fragmented due to mechanical instability.

While the first three cases have considered bulk forms of UN, the next study by Lu et al. used UN films \cite{Lu2016}. To reduce the reactivity of U, which will react when coated with Al or Ti for spent fuel storage, the authors created nitrided U films by sputtering. Films of UN$_{x}$ were prepared at stoichiometries of $x$ = 0.23, 0.68, and 1.66. Their results suggested the formation of an oxynitride, OU$_{x}$N$_{y}$, phase rather than UO$_{2}$ in the UN$_{1.66}$ film. This film formed the thinnest oxide layer, suggesting the greatest resistance to corrosion. The authors reasoned that the poor oxidation resistance of the UN$_{0.68}$ film may be due to the ease with which oxygen can fill a nitrogen vacancy in UN.
\par 
Density functional theory has been used in several studies to probe the atomistic UN corrosion mechanism, notably by the groups Bocharov et al. \cite{Bocharov2013} and Bo et al. \cite{Bo2016}. Bocharov studied oxygen atoms and nitrogen vacancies at the UN (100) and (110) surfaces along with a tilt grain boundary. The Gibbs free energy of defects, such as a nitrogen vacancy discussed above, can be calculated using

\begin{equation}
\Delta G^{N}_{F}(T) = \frac{1}{2} \bigg(E^{UN}_{def}-E^{UN}+2\mu _{N}^{0}(T)\bigg) \quad ,
\end{equation} 
where $E^{UN}_{def}$ is the energy of the surface with the nitrogen vacancy, $E^{UN}$ is the energy of the defect free surface, and $\mu _{N}^{0}$ is the standard nitrogen chemical potential. Bocharov et al. found that an N vacancy can form preferentially at the (110) surface followed by the grain boundary and then the (100) surface, but the oxygen prefered to enter at the grain boundary vacancy site. Regardless, oxygen incorporation yielded negative energy for each case. The study ended with a proposed oxidation mechanism of \cite{Bocharov2013}: 
\begin{enumerate}
	\item chemisorption of O$_{2}$
	\item dissociation of adsorbed O$_{2}$
	\item adsorption of O atop U atoms
	\item high mobility of O along the surface
	\item incorporation of O at N vacancies.
\end{enumerate}
\par 
Bo et al. \cite{Bo2016} extended examination of the UN surface to a full study in the presence of water. Using the (100) surface, Bo et al. studied water coverages ranging from one to four molecules (0.25 to 1 monolayer coverage). They reported the adsorption energies for molecular, partially dissociated, and fully dissociated water, with partially and fully dissociated water being equally favorable. After adsorption, the authors deconstructed the energy levels using local density of states. These plots showed that the valence electrons were localized to the U atoms, while the N electronic densities went to zero starting around 2 eV below the Fermi level.

\section{Discussion}
%reflect on the review
%extrapolate strategies
%how I expect the issue will be resolved
%implications of the solution
%effect on nuclear operations

Across the studies, the experiments show variations in products after subjecting UN to oxygen or steam. Jolkkonen et al. reported the formation of U$_{2}$N$_{3}$, NH$_{3}$, and H$_{2}$. Lopes et al. and Lu et al. both reported oxynitride phases, though the exact stoichiometry is unknown. The results are highly specific to experimental parameters; Lu et al. demonstrated that whether a UO$_{2}$ or  OU$_{x}$N$_{y}$ oxide forms depends on the N concentration, and Jolkkonen et al. showed the products depended on porosity. To better illustrate the differences in experimental parameters, select variables are given in \textbf{Table 1}. With this variance in products, it is impossible to draft a conclusive corrosion mechanism.
\par With DFT, it is possible to precisely control the experimental parameters without requiring a change in setup, but these studies focus on very low oxygen and water coverages. During an accident scenario, it is very unlikely a pristine surface will encounter a single O atom or water molecule. While such results may be useful for feeding into larger scale models that rely on these values for validation,  such models have yet to be created and published. Nevertheless, these single adsorption studies have begun to shed light on why residual N frequently remains and nitrided U resists corrosion: at the Fermi level the electrons are entirely localized to U \cite{Bo2016}. This can be interpreted as a lack of valence N electrons available to react, and thus it acts inert in atmosphere as compared to U. Still, future computational efforts will require examination of greater oxygen and water concentrations, larger surface areas, and a survey of relevant temperatures and pressures. 
\par 
In spite of the lack of a definite mechanism, Lopes et al. \cite{Lopes2017} have demonstrated that U$_{3}$Si$_{2}$ can improve UN sinterability. Similarly, Lu et al. \cite{Lu2016} have shown that nitriding a uranium surface can reduce the thickness of the formed oxide layer.
Mitigating UN corrosion will likely come from a combination of nitriding and adding intermetallics or dopants. First, nitrided pellets may resist corrosion, but if the pellet cracks it will be just as susceptible to degradation as pure UN pellets. Second, intermetallics improve ease of sintering, though whether they do or do not mitigate corrosion remains under debate. An as-of-yet unstudied dopant may prove to mitigate corrosion more successfully than an intermetallic. If UN can at least be made stable in air, it can be used as a Generation IV fuel.
\par 
If UN can be realized as a fuel, its higher thermal conductivity will result in a temperature at the centerline closer to that at the surface, hopefully leading to more uniform pellet expansion and reduced contact with the cladding. If the pressure between cladding and fuel can be reduced, failure of the cladding can be delayed and response time in the event of an accident increased. The nitrides, silicides, or other dopants added to the fuel will need to be accounted for in terms of neutronics, such that the use of control rods can properly be adjusted.

\par 



\begin{tabular}{ |p{3cm}|p{3cm}|p{3cm}|p{3cm}|  }
	\hline
	\multicolumn{4}{|c|}{\textbf{Table 1.} Experimental Parameters} \\
	\hline
	 & Starting Material &Temperature & Pressure\\
	\hline
	Jolkkonen et al. \cite{Jolkkonen2017}   &  UN pellets (77 - 97\%TD) &400 - 425 $^{\circ}$C&  0.05 MPa \\
	Johnson et al. \cite{Johnson2016}   & UN powder ($\approx$20 mg)     &800 $^{\circ}$C&  not reported \\
	Lu et al. \cite{Lu2016}  & UN films    &20 $^{\circ}$C&   UHV \\
	Lopes et al. \cite{Lopes2017}   & UN pellets (95 - 99 \% TD)    & 300 $^{\circ}$C&   9 MPa \\
	\hline
\end{tabular}
\section{Summary}
%reiterate highlights
The high actinide density, high thermal conductivity, and high melting temperature make UN a prospective fuel for accident tolerance in LWRs and for use in Generation IV reactors. However, its chemical instability in an atmosphere of oxygen and water have hindered its implementation. 
\par Jolkkonen et al. \cite{Jolkkonen2017} studied UN pellets under superheated steam to better understand the UN corrosion mechanism. They found better corrosion resistance from denser pellets, U$_{2}$N$_{3}$ formation suggesting intermediary steps for (1), and residual N even after complete oxidation. Johnson et al. \cite{Johnson2016} found UN-U$_{3}$Si$_{2}$ powder oxidized at approximately the same rate as pure, dense UN powder. Lopes et al. \cite{Lopes2017} also found denser pellets reacted less with steam and found grain size contributed to corrosion mechanically. Lu et al. showed higher stoichiometry UN films exhibited thinner oxide layers. Bocharov et al. \cite{Bocharov2013} showed the favorability of oxide incorporation at a nitrogen vacancy and proposed an atomistic reaction mechanism. Lastly, Bo et al. \cite{Bo2016} showed favorable dissociated water adsorption and localization of valence electrons to U.
\par The experiments give insight into how the corrosion reaction changes with respect to experimental parameters, though the exact corrosion mechanism remains elusive. Computational studies have begun to look at single adsorption mechanisms and suggest N does not have active valence electrons. Future computational studies should investigate larger coverarages of adsorbates. 
\par UN corrosion issues may be solved through nitriding and the addition of dopants or intermetallics. If fully realized as a fuel, UN may reduce the severity of accidents and lead to higher burn up. The neutron economy after adjusting UN will need to be carefully considered when added to a reactor.











\bibliography{nuclear}
\bibliographystyle{elsarticle-num}


\end{document}  