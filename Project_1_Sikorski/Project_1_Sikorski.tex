\documentclass[3p,review,11pt]{elsarticle}
\usepackage{lineno,hyperref,notoccite,etoolbox}
\modulolinenumbers[5]
\geometry{margin=1in}
\makeatletter
\def\ps@pprintTitle{%
	\let\@oddhead\@empty
	\let\@evenhead\@empty
	\def\@oddfoot{\centerline{\thepage}}%
	\let\@evenfoot\@oddfoot}
\makeatother
\usepackage{setspace}
\singlespacing
\usepackage{mathptmx}
\usepackage{float,wrapfig}
\begin{document}

\begin{frontmatter}
	\title{Uranium Nitride Corrosion}
	
	\author[boise]{Ember L. Sikorski}
	
	
	\address[boise]{Boise State University}
	
	\begin{abstract}
Abstract stuff
	\end{abstract}	
	
\end{frontmatter}


\section{Introduction}
%introduce topic
%highlight importance of topic to nuclear power production
%materials it affects
%components of the reactor involved

\begin{itemize}
	\item Call for accident tolerant fuel
\end{itemize}

\section{Review}
%current efforts in the area
%may include research on demonstration of issue or mitigation strategies



\section{Discussion}
%reflect on the review
%extrapolate strategies
%how I expect the issue will be resolved
%implications of the solution
%effect on nuclear operations
Arkush and Liu report NO while Jolkonnen does not - environment?
Combination of nitriding, adding dopants, intermetallics
Nitriding reduces interactiong at room temperature, however at higher temperature higher stoichiometric UN decays to UN.
Add intermetallics for ease of fabrication, nitride for room temp handling
\par 
Computational studies at odds with experiment: experiment changes starting conditions, comp changes type of study


\section{Summary}
%reiterate highlights






\cite{Jolkkonen2017}





\section*{References}
CHANGE THE STYLE

\bibliography{nuclear}
\bibliographystyle{elsarticle-num}


\end{document}  