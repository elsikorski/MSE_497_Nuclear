\documentclass[3p,review,11pt]{elsarticle}
\usepackage{lineno,hyperref,notoccite,etoolbox}
\modulolinenumbers[5]
\geometry{margin=1in}
\makeatletter
\def\ps@pprintTitle{%
	\let\@oddhead\@empty
	\let\@evenhead\@empty
	\def\@oddfoot{\centerline{\thepage}}%
	\let\@evenfoot\@oddfoot}
\makeatother
\usepackage{setspace}
\singlespacing
\usepackage{mathptmx}
\usepackage{float,wrapfig}
\begin{document}

\begin{frontmatter}
	\title{Uranium Nitride Corrosion}
	
	\author[boise]{Ember L. Sikorski}
	
	
	\address[boise]{Boise State University}
	
	\begin{abstract}
Abstract stuff
	\end{abstract}	
	
\end{frontmatter}


\section{Introduction}
%introduce topic
%highlight importance of topic to nuclear power production
%materials it affects
%components of the reactor involved

Uranium nitride (UN) is considered a prospective fuel for both light water reactors (LWRs) and Generation IV reactors \cite{Streit2005,Mizutani1998}. The higher fissile density of UN as compared to uranium dioxide benefits fast reactors with their lower neutron cross section \cite{Silva2009}. This higher actinide density additionally benefits LWRs because UN pellets can remain in the reactor longer, leading to longer time between shut downs, and reducing money lost \cite{Lopes2017}. For all reactor types, the high thermal conductivity and high melting temperature make UN an optimal material to resist accidents \cite{Lopes2017}.
\par 
While nuclear power initially developed uranium dioxide for the fuel pellet, the call for accident tolerant fuels (ATF) after Fukushima has drawn attention back to nitrides \cite{Johnson2016}. However, ATF materials must maintain current operational standards as well as improve safety \cite{Zinkle2014}. Despite the beneficial properties of UN, it is unstable the presence of steam or even air \cite{Johnson2016,Jolkkonen2017,Lopes2017}. In the event of cladding failure in a LWR, the pellet will come into contact with steam. As UN degrades, fission products will be released from the fuel matrix, free to interact with the containment structure. Furthermore, nitrogen can react with the steam to form explosive ammonium nitrate \cite{Jolkkonen2017}.
\par 
 In the case of Generation IV reactors, most designs use alternative coolants. However, the instability of UN in air hinders fabrication into a fuel pellet \cite{Lopes2017}. Short of solving UN corrosion at high temperatures and pressures for LWRs, corrosion in air must be mitigated for use as a Generation IV fuel.
 \par 
 To assess the current state of UN corrosion research, this review examines six studies from within the past decade. The first two studies work to better understand UN oxidation in the presence of air \cite{Johnson2016} and steam \cite{Jolkkonen2017}. Next, Lu et al. \cite{Lu2016} and Lopes et al. \cite{Lopes2017} each investigated a method to mitigate UN corrosion: nitriding and introduction of an intermetallic phase, respectively. The last two studies use computational modeling to probe the atomistic initiation of UN corrosion, studying oxygen \cite{Bocharov2013} and water \cite{Bo2016} at UN surfaces.

\section{Review}
%current efforts in the area
%may include research on demonstration of issue or mitigation strategies


\par 
To better understand uranium nitride corrosion, Jolkkonen et al. \cite{Jolkkonen2017} subjected UN pellets to superheated steam. The authors tried varied densities of UN pellets, ranging from 77.6\% to 97.7\% of theoretical density. The lowest density pellet resulted in the greatest amount of degradation, yielding rapid H$_{2}$ production followed by a significant release of N$_{2}$ uncharacteristic of the other, denser pellets. The denser pellets could last up to 90 minutes while subjected to 300 $^{\circ}$C steam, but degraded as the temperature was increased to 400 $^{\circ}$C. The pellets showed different mechanical fracture, with the lowest density pellets turning to powder and the higher density pellets cracking into fragments. This fragmentation would further accelerate corrosion as additional surface is exposed.
\par 
The corrosion mechanism was first proposed as 

\begin{equation}
\mbox{UN} + 2\mbox{H}_{2}\mbox{O} \rightarrow \mbox{UO}_{2} + \mbox{NH}_{3} + \frac{1}{2} \mbox{H}_{2} \quad ,
\end{equation}
suggesting UO$_{2}$ and NH$_{3}$ form in equal proportions and both yield double that of H$_{2}$. However, Jolkkonen et al. found U$_{2}$N$_{3}$ formation, which disrupts the stoichiometry of (1) by reducing NH$_{3}$ formation and increasing H$_{2}$ formation. Additionally, even after hydrolysis finished, residual N$_{2}$ was found in the resultant powder.

Johnson et al. \cite{Johnson2016} studied UN powders and compared the time until oxidation to uranium silicide and uranium dioxide powders. The powders were ramped to 800$^{\circ}$C while the mass was measured with thermogravimetry. At 4.4\% open porosity, UN oxidized at 320 $^{\circ}$C, determined by 5\% mass increase. On the other hand, 0.0\% open porosity UN performed better than even UO$_{2}$, with oxidation onsets of 440 $^{\circ}$C and 405 $^{\circ}$C, respectively. The UN-10\%U$_{2}$Si$_{3}$ powder performed nearly the same as the low porosity UN, with a slightly higher oxidation onset temperature of 450$^{\circ}$C but also slightly higher mass increase due to oxide formation. The authors conclude that while using U$_{3}$Si$_{2}$ in conjunction with UN may improve fabrication of UN, it does not necessarily reduce corrosion.
\par 
Lopes et al. \cite{Lopes2017} took the work of Johnson et al. a step further by fabricating full pellets of UN and UN-10\%U$_{3}$Si$_{2}$. In order to reach the high densities sufficient to reduce oxidation, UN must be sintered at over 2000 $^{\circ}$C. However, this results in accelerated grain growth, creating fast diffusion pathways for both oxygen to enter and fission products to escape. To reduce this necessary temperature, an intermetallic phase of U$_{3}$Si$_{2}$ can be added during sintering.
\par 
Lopes et al. placed the pellets in an autoclave, heated to 300 $^{\circ}$C, and added 9 MPa steam. Like Jolkkonen et al. and Johnson et al., the authors found reduced reactivity with reduced porosity. They found degradation increased in time, and suggested the mechanism depended on the OH and NH$_{3}$ formed during the reaction. The oxygen formed a non-protective layer and larger grains were easily fragmented due to mechanical instability.

While the first three cases have considered bulk forms of UN, the next study by Lu et al. used UN films \cite{Lu2016}. To reduce the reactivity of U, which will react when coated with Al or Ti for spent fuel storage, the authors created nitrided U films by sputtering. Films of UN$_{x}$ were prepared at stoichiometries of $x$ = 0.23, 0.68, and 1.66.
\par 
Density functional theory has been used in several studies to probe the atomistic UN corrosion mechanism, notably by groups Bo et al. \cite{Bo2016} and Bocharov et al. \cite{Bocharov2013}.



\section{Discussion}
%reflect on the review
%extrapolate strategies
%how I expect the issue will be resolved
%implications of the solution
%effect on nuclear operations
Arkush and Liu report NO while Jolkonnen does not - environment?
Combination of nitriding, adding dopants, intermetallics
Nitriding reduces interactiong at room temperature, however at higher temperature higher stoichiometric UN decays to UN.
Add intermetallics for ease of fabrication, nitride for room temp handling
\par 

Bo et al. \cite{Bo2016} used DFT towards determining the initiating of UN corrosion, but while they report optimally water species and adsorption sites, this does little to reveal a reaction mechanism like (1).
Computational studies at odds with experiment: experiment changes starting conditions, comp changes type of study


\begin{tabular}{ |p{3cm}|p{3cm}|p{3cm}|p{3cm}|  }
	%\hline
	%\multicolumn{4}{|c|}{Experimental Parameters} \\
	\hline
	 & Starting Material &Temperature & Pressure\\
	\hline
	Jolkkonen et al. \cite{Jolkkonen2017}   &  UN pellets (77 - 97\%TD) &400 - 425 $^{\circ}$C&  0.05 MPa \\
	Johnson et al. \cite{Johnson2016}   & UN powder ($\approx$20 mg)     &800 $^{\circ}$C&  not reported \\
	Lu et al. \cite{Lu2016}  & UN films    &AFG&   UHV \\
	Lopes et al. \cite{Lopes2017}   & UN pellets (95 - 99 \% TD)    & 300 $^{\circ}$C&   9 MPa \\
	\hline
\end{tabular}
\section{Summary}
%reiterate highlights











\section*{References}


\bibliography{nuclear}
\bibliographystyle{elsarticle-num}


\end{document}  