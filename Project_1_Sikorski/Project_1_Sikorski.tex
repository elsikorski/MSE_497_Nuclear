\documentclass[3p,review,11pt]{elsarticle}
\usepackage{lineno,hyperref,notoccite,etoolbox}
\modulolinenumbers[5]
\geometry{margin=1in}
\makeatletter
\def\ps@pprintTitle{%
	\let\@oddhead\@empty
	\let\@evenhead\@empty
	\def\@oddfoot{\centerline{\thepage}}%
	\let\@evenfoot\@oddfoot}
\makeatother
\usepackage{setspace}
\singlespacing
\usepackage{mathptmx}
\usepackage{float,wrapfig}
\begin{document}

\begin{frontmatter}
	\title{Uranium Nitride Corrosion}
	
	\author[boise]{Ember L. Sikorski}
	
	
	\address[boise]{Boise State University}
	
	\begin{abstract}
Abstract stuff
	\end{abstract}	
	
\end{frontmatter}


\section{Introduction}
%introduce topic
%highlight importance of topic to nuclear power production
%materials it affects
%components of the reactor involved

\begin{itemize}
	\item Call for accident tolerant fuel
\end{itemize}

\section{Review}
%current efforts in the area
%may include research on demonstration of issue or mitigation strategies


To better understand uranium nitride corrosion, Jolkkonen et al. \cite{Jolkkonen2017} and Johnson et al. \cite{Johnson2016} have analyzed UN subjected to steam or air. 

Lu et al. \cite{Lu2016} and Lopes et al. \cite{Lopes2017} each investigated a method to mitigate UN corrosion. 

Density functional theory has been used in several studies to probe the atomistic UN corrosion mechanism, notably by groups Bo et al. \cite{Bo2016} and Bocharov et al. \cite{Bocharov2013}.



\section{Discussion}
%reflect on the review
%extrapolate strategies
%how I expect the issue will be resolved
%implications of the solution
%effect on nuclear operations
Arkush and Liu report NO while Jolkonnen does not - environment?
Combination of nitriding, adding dopants, intermetallics
Nitriding reduces interactiong at room temperature, however at higher temperature higher stoichiometric UN decays to UN.
Add intermetallics for ease of fabrication, nitride for room temp handling
\par 

Bo et al. \cite{Bo2016} used DFT towards determining the initiating of UN corrosion, but while they report optimally water species and adsorption sites, this does little to reveal a reaction mechanism like (1).
Computational studies at odds with experiment: experiment changes starting conditions, comp changes type of study


\begin{tabular}{ |p{3cm}|p{3cm}|p{3cm}|p{3cm}|  }
	%\hline
	%\multicolumn{4}{|c|}{Experimental Parameters} \\
	\hline
	 & Starting Material &Temperature & Pressure\\
	\hline
	Jolkkonen et al. \cite{Jolkkonen2017}   &  UN pellets (77 - 97\%TD) &400 - 425 $^{\circ}$C&  0.05 MPa \\
	Johnson et al. \cite{Johnson2016}   & UN powder ($\approx$20 mg)     &800 $^{\circ}$C&  not reported \\
	Lu et al. \cite{Lu2016}  & UN films    &AFG&   UHV \\
	Lopes et al. \cite{Lopes2017}   & UN pellets (95 - 99 \% TD)    & 300 $^{\circ}$C&   9 MPa \\
	\hline
\end{tabular}
\section{Summary}
%reiterate highlights











\section*{References}


\bibliography{nuclear}
\bibliographystyle{elsarticle-num}


\end{document}  